\documentclass[12pt]{article}
 
\usepackage[margin=1in]{geometry}
\usepackage{amsmath,amsthm,amssymb, mathtools}
\usepackage[T1]{fontenc}
\usepackage{lmodern}
\usepackage{fixltx2e}
\usepackage[shortlabels]{enumitem}
\usepackage{mathrsfs}
\usepackage{kbordermatrix}

\usepackage{graphicx}
\usepackage{bbm}

\renewcommand{\kbldelim}{(}% Left delimiter
\renewcommand{\kbrdelim}{)}% Right delimiter
 
\newcommand{\N}{\mathbb{N}}
\newcommand{\R}{\mathbb{R}}
\newcommand{\Z}{\mathbb{Z}}
\newcommand{\Q}{\mathbb{Q}}
 
\newenvironment{theorem}[2][Theorem]{\begin{trivlist}
\item[\hskip \labelsep {\bfseries #1}\hskip \labelsep {\bfseries #2.}]}{\end{trivlist}}
\newenvironment{lemma}[2][Lemma]{\begin{trivlist}
\item[\hskip \labelsep {\bfseries #1}\hskip \labelsep {\bfseries #2.}]}{\end{trivlist}}
\newenvironment{exercise}[2][Exercise]{\begin{trivlist}
\item[\hskip \labelsep {\bfseries #1}\hskip \labelsep {\bfseries #2.}]}{\end{trivlist}}
\newenvironment{problem}[2][Problem]{\begin{trivlist}
\item[\hskip \labelsep {\bfseries #1}\hskip \labelsep {\bfseries #2.}]}{\end{trivlist}}
\newenvironment{question}[2][Question]{\begin{trivlist}
\item[\hskip \labelsep {\bfseries #1}\hskip \labelsep {\bfseries #2.}]}{\end{trivlist}}
\newenvironment{corollary}[2][Corollary]{\begin{trivlist}
\item[\hskip \labelsep {\bfseries #1}\hskip \labelsep {\bfseries #2.}]}{\end{trivlist}}
\newcommand{\textfrac}[2]{\dfrac{\text{#1}}{\text{#2}}}
\newcommand{\floor}[1]{\left\lfloor #1 \right\rfloor}

\newenvironment{amatrix}[1]{%
  \left(\begin{array}{@{}*{#1}{c}|c@{}}
}{%
  \end{array}\right)
}

\DeclareMathOperator*{\E}{\mathbb{E}}


\begin{document}

\title{Stochastic Processes: Homework 9}

\author{Chris Hayduk}
\date{November 14, 2020}

\maketitle

\begin{problem}{1}
Durrett, Exercise 2.30
\end{problem}

\begin{enumerate}[label=(\alph*)]

\item The total number of calls in an hour is Poisson with mean $4$. Hence, by Theorem 2.11, the number of men calling in an hour is Poisson with mean $4 \cdot (3/4) = 3$ and the number of women calling in an hour is Poisson with mean $4 \cdot (1/4) = 1$. These two Poisson processes are independent as well. Thus, the probability of seeing exactly two men and three women is given by,
\begin{align*}
e^{-3} \cdot \frac{3^2}{2!} \cdot e^{-1} \cdot \frac{1^3}{3!} &\approx 0.014
\end{align*}

\item The sex of the caller is independent of the time of the call, so we can consider this binomial distribution with probability $3/4$ for male and $1/4$ for female. The probability for $3$ males in the first $3$ calls is thus given by,
\begin{align*}
(3/4)^3 &= 27/64\\
&\approx 0.422
\end{align*}

\end{enumerate}
\begin{problem}{2}
Durrett, Exercise 2.33
\end{problem}

\begin{enumerate}[label=\alph*]

\item Given that the customers arrived in the first 5 minutes, the probability that each of them arrived in the first 2 minutes is $2/5$. The arrival times for customers are independent, hence the probability that both customers arrived in the first 2 minutes is given by,
\begin{align*}
(2/5) \cdot (2/5) &= 4/25\\
&= 0.16
\end{align*}

\item The probability that at least $1$ customer arrived in the first $2$ minutes is given by $1$ minus the probability that both arrived in the last $3$ minutes. Each customer has probability $3/5$ of arriving in the last $3$ minutes and, as before, their arrivals are independent. Hence, the probability that at least $1$ arrived in the first $2$ minutes is given by,
\begin{align*}
1 - (3/5)^2 &= 16/25\\
&= 0.64
\end{align*}

\end{enumerate}

\begin{problem}{3}
Durrett, Exercise 2.54
\end{problem}

\begin{problem}{4}
Durrett, Exercise 2.58
\end{problem}


\end{document}