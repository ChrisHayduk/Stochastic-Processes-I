\documentclass[12pt]{article}
 
\usepackage[margin=1in]{geometry}
\usepackage{amsmath,amsthm,amssymb, mathtools}
\usepackage[T1]{fontenc}
\usepackage{lmodern}
\usepackage{fixltx2e}
\usepackage[shortlabels]{enumitem}
\usepackage{mathrsfs}
\usepackage{kbordermatrix}

\usepackage{graphicx}
\usepackage{bbm}

\renewcommand{\kbldelim}{(}% Left delimiter
\renewcommand{\kbrdelim}{)}% Right delimiter
 
\newcommand{\N}{\mathbb{N}}
\newcommand{\R}{\mathbb{R}}
\newcommand{\Z}{\mathbb{Z}}
\newcommand{\Q}{\mathbb{Q}}
 
\newenvironment{theorem}[2][Theorem]{\begin{trivlist}
\item[\hskip \labelsep {\bfseries #1}\hskip \labelsep {\bfseries #2.}]}{\end{trivlist}}
\newenvironment{lemma}[2][Lemma]{\begin{trivlist}
\item[\hskip \labelsep {\bfseries #1}\hskip \labelsep {\bfseries #2.}]}{\end{trivlist}}
\newenvironment{exercise}[2][Exercise]{\begin{trivlist}
\item[\hskip \labelsep {\bfseries #1}\hskip \labelsep {\bfseries #2.}]}{\end{trivlist}}
\newenvironment{problem}[2][Problem]{\begin{trivlist}
\item[\hskip \labelsep {\bfseries #1}\hskip \labelsep {\bfseries #2.}]}{\end{trivlist}}
\newenvironment{question}[2][Question]{\begin{trivlist}
\item[\hskip \labelsep {\bfseries #1}\hskip \labelsep {\bfseries #2.}]}{\end{trivlist}}
\newenvironment{corollary}[2][Corollary]{\begin{trivlist}
\item[\hskip \labelsep {\bfseries #1}\hskip \labelsep {\bfseries #2.}]}{\end{trivlist}}
\newcommand{\textfrac}[2]{\dfrac{\text{#1}}{\text{#2}}}
\newcommand{\floor}[1]{\left\lfloor #1 \right\rfloor}

\newenvironment{amatrix}[1]{%
  \left(\begin{array}{@{}*{#1}{c}|c@{}}
}{%
  \end{array}\right)
}

\DeclareMathOperator*{\E}{\mathbb{E}}


\begin{document}

\title{Stochastic Processes: Homework 6}

\author{Chris Hayduk}
\date{October 14, 2020}

\maketitle

\begin{problem}{1}
Durrett, Exercise 1.51
\end{problem}

We have the following transition matrix,


\begin{align*}
p = \kbordermatrix{
    & HHH & HHT & HTH & THH & HTT & THT & TTH & TTT\\
    HHH & 1/2 & 1/2 & 0 & 0 & 0 & 0 & 0 & 0\\
    HHT & 0 & 0 & 1/2 & 0 & 1/2 & 0 & 0 & 0\\
    HTH & 0 & 0 & 0 & 1/2 & 0 & 1/2 & 0 & 0\\
    THH & 1/2 & 1/2 & 0 & 0 & 0 & 0 & 0 & 0\\
    HTT & 0 & 0 & 0 & 0 & 0 & 0 & 1/2 & 1/2\\
    THT & 0 & 0 & 1/2 & 0 & 1/2 & 0 & 0 & 0\\
    TTH & 0 & 0 & 0 & 1/2 & 0 & 1/2 & 0 & 0\\
    TTT & 0 & 0 & 0 & 0 & 0 & 0 & 1/2 & 1/2
  }
\end{align*}



\begin{problem}{2}
Durrett, Exercise 1.52
\end{problem}

\begin{enumerate}[label=(\alph*)]

\item Larry starts with $1$ coupon. That is, $X_0 = 1$. We also have the following transition probability:

\begin{align*}
p = \kbordermatrix{
    & 0 & 1 & 2 & 3 \\
    0 & 1 & 0 & 0 & 0\\
    1 & 0.5 & 0 & 0.5 & 0\\
    2 & 0 & 0.5 & 0 & 0.5\\
    3 & 0 & 0 & 0 & 1\\
  }
\end{align*}

We have $h(0) = 0$ and $h(3) = 1$ with the remaining finite set of states $\{1, 2\}$. Hence, we can apply Theorem 1.28, which gives us $h(x) = \sum_y p(x, y) h(y)$. In addition, since $P_x(V_{\{0\}} \wedge V_{\{3\}} < \infty) > 0$ for $x = 1, 2$, we also have that $h(x) = P_x(V_3 < V_0)$. We have $q = p = 0.5$, so for $0 < x < 3$, we get,
\begin{align*}
h(x) = 0.5h(x+1) + 0.5h(x-1)
\end{align*}

We get,
\begin{align*}
h(1) &= 0.5h(2)\\
h(2) &= 0.5 + 0.5h(1)
\end{align*}

Combining equation $(1)$ and $(2)$ yields $h(2) = 0.5/0.75 \approx 0.667$. Hence, we have,
\begin{align*}
h(1) = P_1(V_3 < V_0) &= \frac{0.25}{0.75}\\
\approx 0.333
\end{align*}

\item We have the following system of equations,
\begin{align*}
g(1) &= 1 + 0.5g(2)\\
g(2) &= 1 + 0.5g(1)
\end{align*}

So we get that $g(2) = 1 + 0.5 + 0.25g(2) = 2$ and thus $g(1) = 1 + 0.5(2) = 2$ as well.\\

Hence, starting from $1$ ticket, Larry will need $2$ plays on average in order to win or lose the game.

\end{enumerate}

\begin{problem}{3}
Durrett, Exercise 1.67
\end{problem}

\begin{enumerate}[label=(\alph*)]

\item Observe that for $X_n = n$ with $0 \leq n \leq 5$, we there are $n$ sides of the die which have already been seen and $6 - n$ sides which have not been seen. Since each side is equally probable to show up, we have that,
\begin{align*}
P(X_{n+1} = n) &= n/6\\
P(X_{n+1} = n + 1) &= (6-n)/6
\end{align*}

When $X_n = 6$, we have $P(X_{n+1} = 6) = 1$.\\

Thus, the transition probability matrix is,

\begin{align*}
p &= \kbordermatrix{
    & 0 & 1 & 2 & 3 & 4 & 5 & 6\\
    0 & 0 & 1 & 0 & 0 & 0 & 0 & 0\\
    1 & 0 & 1/6 & 5/6 & 0 & 0 & 0 & 0\\
    2 & 0 & 0 & 2/6 & 4/6 & 0 & 0 & 0\\
    3 & 0 & 0 & 0 & 3/6 & 3/6 & 0 & 0\\
    4 & 0 & 0 & 0 & 0 & 4/6 & 2/6 & 0\\
    5 & 0 & 0 & 0 & 0 & 0 & 5/6 & 1/6\\
    6 & 0 & 0 & 0 & 0 & 0 & 0 & 1
  }
\end{align*}

\item Let $T = \min \{n: X_n = 6\}$. We need to find $ET$. That is, $ET$ the expected minimum number of rolls to see all $6$ sides of the die.\\

Consider the matrix after deleting the row and column corresponding to $6$:
\begin{align*}
r &= \kbordermatrix{
    & 0 & 1 & 2 & 3 & 4 & 5\\
    0 & 0 & 1 & 0 & 0 & 0 & 0\\
    1 & 0 & 1/6 & 5/6 & 0 & 0 & 0\\
    2 & 0 & 0 & 2/6 & 4/6 & 0 & 0\\
    3 & 0 & 0 & 0 & 3/6 & 3/6 & 0\\
    4 & 0 & 0 & 0 & 0 & 4/6 & 2/6\\
    5 & 0 & 0 & 0 & 0 & 0 & 5/6\\
  }
\end{align*}

So we have,
\begin{align*}
I - r = \kbordermatrix{
    & 0 & 1 & 2 & 3 & 4 & 5\\
    0 & 1 & -1 & 0 & 0 & 0 & 0\\
    1 & 0 & 5/6 & -5/6 & 0 & 0 & 0\\
    2 & 0 & 0 & 4/6 & -4/6 & 0 & 0\\
    3 & 0 & 0 & 0 & 3/6 & -3/6 & 0\\
    4 & 0 & 0 & 0 & 0 & 2/6 & -2/6\\
    5 & 0 & 0 & 0 & 0 & 0 & 1/6\\
  }
\end{align*}

This yields,
\begin{align*}
(I - r)^{-1} &= \kbordermatrix{
    & 0 & 1 & 2 & 3 & 4 & 5\\
    0 & 1 & 6/5 & 3/2 & 2 & 3 & 6\\
    1 & 0 & 6/5 & 3/2 & 2 & 3 & 6\\
    2 & 0 & 0 & 3/2 & 2 & 3 & 6\\
    3 & 0 & 0 & 0 & 2 & 3 & 6\\
    4 & 0 & 0 & 0 & 0 & 3 & 6\\
    5 & 0 & 0 & 0 & 0 & 0 & 6\\
  }
\end{align*}

Lastly, we have
\begin{align*}
(I - r)^{-1} \mathbbm{1} &= \begin{pmatrix}
14.7\\
13.7\\
12.5\\
11\\
9\\
67
\end{pmatrix}
\end{align*}

Thus, starting from state $0$, we would expect to make $14.7$ moves on average before seeing all $6$ sides of the die.
\end{enumerate}

\end{document}