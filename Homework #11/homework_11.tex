\documentclass[12pt]{article}
 
\usepackage[margin=1in]{geometry}
\usepackage{amsmath,amsthm,amssymb, mathtools}
\usepackage[T1]{fontenc}
\usepackage{lmodern}
\usepackage{fixltx2e}
\usepackage[shortlabels]{enumitem}
\usepackage{mathrsfs}
\usepackage{kbordermatrix}

\usepackage{graphicx}
\usepackage{bbm}

\renewcommand{\kbldelim}{(}% Left delimiter
\renewcommand{\kbrdelim}{)}% Right delimiter
 
\newcommand{\N}{\mathbb{N}}
\newcommand{\R}{\mathbb{R}}
\newcommand{\Z}{\mathbb{Z}}
\newcommand{\Q}{\mathbb{Q}}
 
\newenvironment{theorem}[2][Theorem]{\begin{trivlist}
\item[\hskip \labelsep {\bfseries #1}\hskip \labelsep {\bfseries #2.}]}{\end{trivlist}}
\newenvironment{lemma}[2][Lemma]{\begin{trivlist}
\item[\hskip \labelsep {\bfseries #1}\hskip \labelsep {\bfseries #2.}]}{\end{trivlist}}
\newenvironment{exercise}[2][Exercise]{\begin{trivlist}
\item[\hskip \labelsep {\bfseries #1}\hskip \labelsep {\bfseries #2.}]}{\end{trivlist}}
\newenvironment{problem}[2][Problem]{\begin{trivlist}
\item[\hskip \labelsep {\bfseries #1}\hskip \labelsep {\bfseries #2.}]}{\end{trivlist}}
\newenvironment{question}[2][Question]{\begin{trivlist}
\item[\hskip \labelsep {\bfseries #1}\hskip \labelsep {\bfseries #2.}]}{\end{trivlist}}
\newenvironment{corollary}[2][Corollary]{\begin{trivlist}
\item[\hskip \labelsep {\bfseries #1}\hskip \labelsep {\bfseries #2.}]}{\end{trivlist}}
\newcommand{\textfrac}[2]{\dfrac{\text{#1}}{\text{#2}}}
\newcommand{\floor}[1]{\left\lfloor #1 \right\rfloor}

\newenvironment{amatrix}[1]{%
  \left(\begin{array}{@{}*{#1}{c}|c@{}}
}{%
  \end{array}\right)
}

\DeclareMathOperator*{\E}{\mathbb{E}}


\begin{document}

\title{Stochastic Processes: Homework 11}

\author{Chris Hayduk}
\date{December 4, 2020}

\maketitle

\begin{problem}{1}
Durrett, Exercise 3.4
\end{problem}

Let $g_i$ be the random fair from distribution $G$ paid at time $i$. Then we have that $W(t) = \sum_{i = 1}^{N(t)} g_i$. So
\begin{align*}
E W(t) &= E\left[\sum_{i = 1}^{N(t)} g_i\right]\\
&= \sum_{i = 1}^{N(t)} E g_i\\
&= \sum_{i = 1}^{N(t)} \mu_G\\
&= N(t) \cdot \mu_G
\end{align*}

So we get,
\begin{align*}
\lim_{t \to \infty} EW(t)/t &= \mu_G \lim_{t \to \infty} N(t)/t
\end{align*}

Then, using Theorem 3.1, we get,
\begin{align*}
\mu_G \lim_{t \to \infty} N(t)/t &= \mu_G \cdot 1/\mu_F\\
&= \frac{\mu_G}{\mu_F}
\end{align*}

\begin{problem}{2}
Durrett, Exercise 3.11
\end{problem}

\begin{enumerate}[label=\alph*)]

\item

\item

\item

\end{enumerate}

\begin{problem}{3}
Durrett, Exercise 3.17
\end{problem}

Let $w_i$ denote the time that the $i$-th person waits until the start of the tour (in minutes). Note that we need $k$ people to start the tour, that each tour costs \$20, and that it costs \$0.10 per minute that each person waits. Hence, we have that the cost per person can be formulated as,
\begin{align*}
f(k) = \frac{0.1 \cdot (w_1 + w_2 + \cdots + w_k) + 20}{k}
\end{align*}

To find the average, let us take the expectation:,
\begin{align*}
E \left[\frac{0.1 \cdot (w_1 + w_2 + \cdots + w_k) + 20}{k}\right] &= \frac{0.1 \cdot (Ew_1 + Ew_2 + \cdots + Ew_k) + 20}{k}
\end{align*}

Since people arrive at a rate of 1 per minute, we get the following expected waiting times,
\begin{align*}
Ew_i = k - i
\end{align*}

This yields,
\begin{align*}
\frac{0.1 \cdot (Ew_1 + Ew_2 + \cdots + Ew_k) + 20}{k} &= \frac{0.1 \cdot 1/2 \cdot (k-1)k + 20}{k}\\
&= \frac{0.05 \cdot (k-1)k + 20}{k}\\
&= \frac{0.05k^2 - 0.05k + 20}{k}
\end{align*}

Now we want to minimize this equation. Let us take the derivative with respect to $k$ and set it to $0$, yielding,
\begin{align*}
0 = 0.05 - \frac{20}{k^2}
\end{align*}

This gives $k = 20$ or $k = -20$. We can't have a negative number of people, so our only valid critical point is $k = 20$. We can now use the second derivative test to ensure that this is a local minimum:

\begin{align*}
\frac{d}{dk} 0.05 - \frac{20}{k^2} = \frac{40}{k^3}
\end{align*} 

So $f''(20) = 40/20^3 > 0$, so $20$ is a local minimum as required.

\begin{problem}{4}
\end{problem}

\begin{enumerate}[label=\Alph*)]

\item

\item

\item

\end{enumerate}



\end{document}