\documentclass[12pt]{article}
 
\usepackage[margin=1in]{geometry}
\usepackage{amsmath,amsthm,amssymb, mathtools}
\usepackage[T1]{fontenc}
\usepackage{lmodern}
\usepackage{fixltx2e}
\usepackage[shortlabels]{enumitem}
\usepackage{mathrsfs}
 
\newcommand{\N}{\mathbb{N}}
\newcommand{\R}{\mathbb{R}}
\newcommand{\Z}{\mathbb{Z}}
\newcommand{\Q}{\mathbb{Q}}
 
\newenvironment{theorem}[2][Theorem]{\begin{trivlist}
\item[\hskip \labelsep {\bfseries #1}\hskip \labelsep {\bfseries #2.}]}{\end{trivlist}}
\newenvironment{lemma}[2][Lemma]{\begin{trivlist}
\item[\hskip \labelsep {\bfseries #1}\hskip \labelsep {\bfseries #2.}]}{\end{trivlist}}
\newenvironment{exercise}[2][Exercise]{\begin{trivlist}
\item[\hskip \labelsep {\bfseries #1}\hskip \labelsep {\bfseries #2.}]}{\end{trivlist}}
\newenvironment{problem}[2][Problem]{\begin{trivlist}
\item[\hskip \labelsep {\bfseries #1}\hskip \labelsep {\bfseries #2.}]}{\end{trivlist}}
\newenvironment{question}[2][Question]{\begin{trivlist}
\item[\hskip \labelsep {\bfseries #1}\hskip \labelsep {\bfseries #2.}]}{\end{trivlist}}
\newenvironment{corollary}[2][Corollary]{\begin{trivlist}
\item[\hskip \labelsep {\bfseries #1}\hskip \labelsep {\bfseries #2.}]}{\end{trivlist}}
\newcommand{\textfrac}[2]{\dfrac{\text{#1}}{\text{#2}}}
\newcommand{\floor}[1]{\left\lfloor #1 \right\rfloor}

\DeclareMathOperator*{\E}{\mathbb{E}}


\begin{document}

\title{Stochastic Processes: Homework 1}

\author{Chris Hayduk}
\date{September 2, 2020}

\maketitle

\begin{problem}{1}
\end{problem}

Let $A_1 = $ ``Donna drew a fair coin'', $A_2 = $ ``Donna drew the two-headed coin'' and let $B = $ ``Donna flipped 6 heads in a row''.\\

Note that

\begin{align*}
P(A_1) &= \frac{64}{65}\\
P(B|A_1) &= \frac{1/2^6}{64/65} = \frac{65}{4096} \approx 0.01587
\end{align*}

and

\begin{align*}
P(A_2) &= \frac{1}{65}\\
P(B|A_2) &= 1
\end{align*}

Hence,
\begin{align*}
\sum P(A_j) P(B | A_j) &= \frac{64}{65} \cdot \frac{65}{4096} + \frac{1}{65} \cdot 1\\
&= \frac{129}{4160} \approx 0.031
\end{align*}

Thus, we have,
\begin{align*}
P(A_2 | B) &= \frac{P(A_2)P(B|A_2)}{\sum P(A_j) P(B | A_j)}\\
&= \frac{1/65}{129/4160}\\
&= \frac{64}{129} \approx 0.4961
\end{align*}

So if Donna flips 6 heads in a row, there is about a 49.61\% chance that she chose the two-headed coin.

\newpage

\begin{problem}{2}
\end{problem}

We have that $X_1, X_2$ are two independent uniform (0, 1) random variables on the same probability space. In addition, we have that $Y = \min \{X_1, X_2\}$.\\

In order to find the CDF of $Y$, we must find $P(Y \leq y)$. Note that,

\begin{align*}
P(Y \leq y) = P(\min \{X_1, X_2\} \leq y)
\end{align*}

In addition, note that,
\begin{align*}
P(Y \leq y) &= 1 - P(Y > y)\\
&= 1 - P(\min \{X_1, X_2\} > y)
\end{align*}

Since we are taking a minimum, $\min \{X_1, X_2\} > y$ if and only if we have both $X_1 > y$ and $X_2 > y$. Since $X_1$ and $X_2$ are independent, we can separate this joint probability as follows:

\begin{align*}
P(Y \leq y) &= 1 - P(Y > y)\\
&= 1 - P(\min \{X_1, X_2\} > y)\\
&= 1 - P(X_1 > y) \cdot P(X_2 > y)\\
&= 1 - P(X_1 > y)^2\\
&= 1 - (1 - P(X_1 \leq y))^2
\end{align*}

We know from p. 258 in the text that for a uniform random variable $X$, we have $P(X \leq x) = (x - a)/(b - a)$. Thus, in our case, we have
\begin{align*}
1 - P(X_1 \leq y) &= 1 - (y - 0)/(1 - 0)\\
&= 1 - x
\end{align*}

Hence,
\begin{align*}
P(Y \leq y) = 1 - (1 - y)^2
\end{align*}

and as a result, the CDF of Y is,
\begin{align*}
F_Y(y) = \begin{cases} 
      0 & y \leq 0 \\
      1 - (1 - y)^2 & 0 < y < 1 \\
      1 & y \geq 1
   \end{cases}
\end{align*}

\newpage

\begin{problem}{3}
\end{problem}

Let $X$ and $Y$ be independent, integer-valued random variables and fix $n \in \mathbb{Z}$. Then we have,
\begin{align*}
P(X + Y = n) &= \sum_{m \in \mathbb{Z}} P(X = m, Y = n - m)\\
&= \sum_{m \in \mathbb{Z}} P(Y = n - m  \, | \, X = m) P(X = m)
\end{align*}

Since $X$ and $Y$ are independent, we have
\begin{align*}
P(X + Y = n) &= \sum_{m \in \mathbb{Z}} P(Y = n - m  \, | \, X = m) P(X = m)\\
&= \sum_{m \in \mathbb{Z}} P(X = m) P(Y = n - m)
\end{align*}
\begin{problem}{4}
\end{problem}

\begin{enumerate}[(\Alph*)]

\item

We have

\begin{align*}
\E[YZ] &= \int_{0}^1 \sin(2\pi x) \cdot \cos(2\pi x) dx\\
&= \int_{0}^1 \frac{1}{2} \sin(4\pi x) dx\\
&= \frac{1}{2} \int_{0}^1 \sin(4\pi x) dx\\
&= \frac{1}{2} \left[ - \frac{\cos(4\pi \cdot 1)}{4 \pi} - \left(- \frac{\cos(4\pi \cdot 0)}{4 \pi}\right)\right]\\
&= \frac{1}{2} \left[ - \frac{\cos(4\pi)}{4 \pi} + \frac{\cos(0)}{4 \pi}\right]\\
&= \frac{1}{2} \left[ - \frac{1}{4 \pi} + \frac{1}{4 \pi}\right]\\
&= 0
\end{align*}

and

\begin{align*}
\E[Y] \cdot \E[Z] &= \int_0^1 \sin(2\pi x) dx \cdot \int_0^1 \cos(2\pi x) dx\\
&= \left[-\frac{\cos(2\pi)}{2\pi} + \frac{\cos(0)}{2\pi}\right] \cdot \left[\frac{\sin(2\pi)}{2\pi} - \frac{\sin(0)}{2\pi}\right]\\
&= \left[-\frac{1}{2\pi} + \frac{1}{2\pi}\right] \cdot \left[\frac{0}{2\pi} - \frac{0}{2\pi}\right]\\
&= 0 \cdot 0 = 0
\end{align*}

Hence, $\E[YZ] = \E[Y] \cdot \E[Z]$ and, as a result, we have that $Y$ and $Z$ are uncorrelated.

\newpage
\item

$Y$ and $Z$ are not independent variables. To show this, we need to find sets $A$ and $B$ such that

\begin{align*}
P(Y \in A, Z \in B) \neq P(Y \in A) \cdot P(Z \in B)
\end{align*}

Let $A = B = [0.9, 1]$.

Then,

\begin{align*}
P(Y \in A, Z \in B) &= P(0.9 \leq Y \leq 1, 0.9 \leq Z \leq 1)\\
&= \int_0.9^1 \sin(2\pi x) \cdot \cos(2\pi x) dx\\
&= \frac{1}{2} \int_{0.9}^1 \sin(4\pi x) dx\\
&= \frac{1}{2} \left[-\frac{\cos(4\pi)}{4\pi} - \frac{-\cos(4\pi \cdot 0.9}{4\pi}\right]\\
&\approx -0.02749
\end{align*}

However,
\begin{align*}
P(Y \in A) \cdot P(Z \in B) &= P(0.9 \leq Y \leq 1) \cdot P(0.9 \leq Z \leq 1)\\
&= \int_{0.9}^1 \sin(2\pi x) dx \cdot \int_{0.9}^1 \cos(2\pi x) dx\\
&\approx -0.00284
\end{align*}

Hence, $P(Y \in A, Z \in B) \neq P(Y \in A) \cdot P(Z \in B)$ and $Y$ and $Z$ are not independent.

\end{enumerate}

\end{document}