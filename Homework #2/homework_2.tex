\documentclass[12pt]{article}
 
\usepackage[margin=1in]{geometry}
\usepackage{amsmath,amsthm,amssymb, mathtools}
\usepackage[T1]{fontenc}
\usepackage{lmodern}
\usepackage{fixltx2e}
\usepackage[shortlabels]{enumitem}
\usepackage{mathrsfs}
\usepackage{kbordermatrix}

\renewcommand{\kbldelim}{(}% Left delimiter
\renewcommand{\kbrdelim}{)}% Right delimiter
 
\newcommand{\N}{\mathbb{N}}
\newcommand{\R}{\mathbb{R}}
\newcommand{\Z}{\mathbb{Z}}
\newcommand{\Q}{\mathbb{Q}}
 
\newenvironment{theorem}[2][Theorem]{\begin{trivlist}
\item[\hskip \labelsep {\bfseries #1}\hskip \labelsep {\bfseries #2.}]}{\end{trivlist}}
\newenvironment{lemma}[2][Lemma]{\begin{trivlist}
\item[\hskip \labelsep {\bfseries #1}\hskip \labelsep {\bfseries #2.}]}{\end{trivlist}}
\newenvironment{exercise}[2][Exercise]{\begin{trivlist}
\item[\hskip \labelsep {\bfseries #1}\hskip \labelsep {\bfseries #2.}]}{\end{trivlist}}
\newenvironment{problem}[2][Problem]{\begin{trivlist}
\item[\hskip \labelsep {\bfseries #1}\hskip \labelsep {\bfseries #2.}]}{\end{trivlist}}
\newenvironment{question}[2][Question]{\begin{trivlist}
\item[\hskip \labelsep {\bfseries #1}\hskip \labelsep {\bfseries #2.}]}{\end{trivlist}}
\newenvironment{corollary}[2][Corollary]{\begin{trivlist}
\item[\hskip \labelsep {\bfseries #1}\hskip \labelsep {\bfseries #2.}]}{\end{trivlist}}
\newcommand{\textfrac}[2]{\dfrac{\text{#1}}{\text{#2}}}
\newcommand{\floor}[1]{\left\lfloor #1 \right\rfloor}

\DeclareMathOperator*{\E}{\mathbb{E}}


\begin{document}

\title{Stochastic Processes: Homework 2}

\author{Chris Hayduk}
\date{September 9, 2020}

\maketitle

\begin{problem}{1}
Durrett, Exercise 1.2
\end{problem}

Observe that if $0 < X_n < 5$, then there are 3 possibilities for the transition:
\begin{enumerate}
\item $X_{n+1} = X_n$. This can occur if two white balls are exchanged or if two black balls are exchanged
\item $X_{n+1} = X_n + 1$. This occurs if a black ball from the left urn is exchanged for a white ball from the right urn
\item $X_{n+1} = X_n - 1$. This occurs if a white ball from the left urn is exchanged for a black ball from the right urn
\end{enumerate}

If $X_n = 0$, then there is only one possible transition: $X_{n+1} = X_n + 1$.\\

If $X_n = 5$, then there is only one possible transition: $X_{n+1} = X_n - 1$.\\

From these facts, we can derive the transition probability matrix:
\begin{align*}
\kbordermatrix{
    & 0 & 1 & 2 & 3 & 4 & 5 \\
    0 & 0 & 1 & 0 & 0 & 0 & 0 \\
    1 & 0.04 & 0.32 & 0.64 & 0 & 0 & 0 \\
    2 & 0 & 0.16 & 0.48 & 0.36 & 0 & 0\\
    3 & 0 & 0 & 0.36 & 0.48 & 0.16 & 0 \\
    4 & 0 & 0 & 0 & 0.64 & 0.32 & 0.04\\
    5 & 0 & 0 & 0 & 0 & 1 & 0
  }
\end{align*}

Let $x = $ number of white balls in left urn and $y = $ number of white balls in right urn. Hence $5-x = $ number of black balls in left urn and $5-y = $ number of black balls in right urn.\\

Then for $0 < i < 5$, we have
\begin{align*}
p(i, i-1) &= \frac{x}{5} \cdot \frac{5-y}{5}\\
p(i, i) &= \frac{x}{5} \cdot \frac{y}{5} + \frac{5-x}{5} \cdot \frac{5-y}{5}\\
p(i, i+1) &= \frac{5-x}{5} \cdot \frac{y}{5}
\end{align*}

\newpage
\begin{problem}{2}
Durrett, Exercise 1.3
\end{problem}

Note that $X_n$ can be any integer from 0 to 5. Hence, the state space is $\{0, 1, 2, 3, 4, 5\}$.\\

For each individual roll $Y_k$, the possible sums are as follows, along with possible combinations to get that sum:
\begin{align*}
2:& (1, 1)\\
3:& (2, 1), (1, 2)\\
4:& (2, 2), (1, 3), (3, 1)\\
5:& (1, 4), (4, 1), (3, 2), (2, 3)\\
6:& (3, 3), (4, 2), (2, 4)\\
7:& (4, 3), (3, 4)\\
8:& (4, 4)
\end{align*}

We see that there are $16$ possible rolls with this pair of dice. Using the possible outcomes listed above, we can derive probabilities for each sum:
\begin{align*}
P(Y_k = 2) &= 0.0625\\
P(Y_k = 3) &= 0.125\\
P(Y_k = 4) &= 0.1875\\
P(Y_k = 5) &= 0.25\\
P(Y_k = 6) &= 0.1875\\
P(Y_k = 7) &= 0.125\\
P(Y_k = 8) &= 0.0625
\end{align*}

Now we can show which congruence classes these sums to modulo 6:
\begin{align*}
2 &= \bar{2}\\
3 &= \bar{3}\\
4 &= \bar{4}\\
5 &= \bar{5}\\
6 &= \bar{0}\\
7 &= \bar{1}\\
8 &= \bar{2}\\
\end{align*}

Now we can convert the probabilities for $Y_k$ using these congruence classes,
\begin{align*}
P(Y_k = \bar{0}) &= 0.1875\\
P(Y_k = \bar{1}) &= 0.125 + 0.125 = 0.25\\
P(Y_k = \bar{2}) &= 0.0625 + 0.0625 = 0.125\\
P(Y_k = \bar{3}) &= 0.125\\
P(Y_k = \bar{4}) &= 0.1875\\
P(Y_k = \bar{5}) &= 0.25\\
\end{align*}

Now using these probabilities and the properties of modular arithmetic, we will derive a transition probability matrix for $X_n$:
\begin{align*}
\kbordermatrix{
    & \bar{0} & \bar{1} & \bar{2} & \bar{3} & \bar{4} & \bar{5} \\
    \bar{0} & 0.1875 & 0.25 & 0.125 & 0.125 & 0.1875 & 0.25 \\
    \bar{1} & 0.25 & 0.1875 & 0.25 & 0.125 & 0.125 & 0.1875 \\
    \bar{2} & 0.1875 & 0.25 & 0.1875 & 0.25 & 0.125 & 0.125\\
    \bar{3} & 0.125 & 0.1875 & 0.25 & 0.1875 & 0.25 & 0.125 \\
    \bar{4} & 0.125 & 0.125 & 0.1875 & 0.25 & 0.1875 & 0.25\\
    \bar{5} & 0.25 & 0.125 & 0.125 & 0.1875 & 0.25 & 0.1875
  }
\end{align*}

So if $j \geq i$, we have 
\begin{align*}
p(i, j) = P(Y_{k} = \overline{j-i})
\end{align*}

\newpage
\begin{problem}{3}
Durrett, Exercise 1.58
\end{problem}

\newpage
\begin{problem}{4}
\end{problem}
\end{document}